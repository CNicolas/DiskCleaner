\documentclass{article}
\usepackage[utf8]{inputenc}
\usepackage{listings}
\usepackage{color}
\usepackage[a4paper]{geometry}

\definecolor{dkgreen}{rgb}{0,0.6,0}
\definecolor{gray}{rgb}{0.5,0.5,0.5}
\definecolor{mauve}{rgb}{0.58,0,0.82}

\lstset{frame=tb,
  language=C,
  showstringspaces=false,
  columns=flexible,
  basicstyle={\small\ttfamily},
  numbers=none,
  numberstyle=\tiny\color{gray},
  keywordstyle=\color{blue},
  commentstyle=\color{dkgreen},
  stringstyle=\color{mauve},
  breaklines=true,
  breakatwhitespace=true
  tabsize=3
}

\begin{document}
\title{DiskCleaner - A Windows service}
\author{Clément Nicolas}
\maketitle

\section{Project}
DiskCleaner is a project for ETNA school.
The aim is to install a Windows service that cleans a directory, which path is stored in a registry key.
\subsection{Technology used}
For this, we use the Windows API.

%----------------------------------------------------------------------------------------------------------------------------------------------------
\newpage
\section{ReadMe}
The file with the logs of the last call is ``C:\textbackslash{}AutoClean.log''
\\
Three commands are available :
\subsection{Commands}

	\begin{itemize}
		\item \textbf{setpath} : AutoClean.exe -p path\_to\_clean
	\end{itemize}
	It sets the path to clean in the register, for example if you put ``D:\textbackslash{}Users\textbackslash{}me\textbackslash{}test'', this directory will be cleaned
	\begin{itemize}
		\item \textbf{install} : AutoClean.exe -i
	\end{itemize}
	It installs the service on you system. It is now available in your services.msc
	\begin{itemize}
		\item \textbf{delete} : AutoClean.exe -d
	\end{itemize}
	It deletes the service of your system. It will disappear from you services.msc
		
\subsection{Install The Program}
\begin{itemize}
	\item Run the \textbf{AutoClean\_setup.exe}
\end{itemize}

\subsection{Install The Service}
\begin{itemize}
	\item Go into the setup directory, and run a cmd.exe with administrator rights
	\item Call the \textbf{setpath} and \textbf{install} (the order doesn't matter).
\end{itemize}

\subsection{Run AutoClean}
\begin{itemize}
	\item Go in the services.msc
	\item Find the description \textbf{Service de nettoyage ETNA}
	\item Click on the \textbf{start} button, or right-click -- start
	\item Then you can stop by clicking on the \textbf{stop} button, or right-click -- stop
	\item The directory you entered is cleaned recursively !
\end{itemize}

\subsection{Delete The Service}
\begin{itemize}
	\item Call the \textbf{delete} command to remove the service.
\end{itemize}

\subsection{Uninstall The Programm}
\begin{itemize}
	\item In the setup directory, simply run uninst*.exe. Or use the configuration panel, or any utility you know.
\end{itemize}

%----------------------------------------------------------------------------------------------------------------------------------------------------

\newgeometry{left=3cm,bottom=0.1cm}

\newpage
\section{Functions}
% LE MAIN
\subsection{Main}
\paragraph{}
The main function.

\begin{lstlisting}
int	main(int argc, char* argv[])
{
	BOOL logfileready = TRUE;
	SERVICE_TABLE_ENTRY table[] = { { SERVICE_NAME, ServiceMain }, { NULL, NULL } };

	g_LogFile = CreateFile(SERVICE_LOG_FILE, GENERIC_WRITE, 0, NULL, CREATE_ALWAYS, FILE_ATTRIBUTE_NORMAL, NULL);
	if (g_LogFile == INVALID_HANDLE_VALUE)
	{
		logfileready = FALSE;
		fprintf(stderr, "Main => CreateFile() failed with : %d\nThe service won't log correctly...\n", GetLastError());
	}

	if (argc > 1)
	{
		if (strcmp(argv[1], "-i") == 0)
			return (launchInstall(logfileready));
		else if ((strcmp(argv[1], "-p") == 0) && (argc > 2))
			return (launchSetPath(argv[2], logfileready));
		else if (strcmp(argv[1], "-d") == 0)
			return (launchDelete(logfileready));
		else
		{
			fprintf(stderr, "Bad argument. Expected :\n\t[-i] to install\n\t[-d] to delete\n\t[-p path] to set a path to clean");
			return (EXIT_FAILURE);
		}
	}
	else
		StartServiceCtrlDispatcher(table);

	if (!CloseHandle(g_LogFile))
		fprintf(stderr, "Main => CloseHandle failed with : %d\n", GetLastError());

	return (EXIT_SUCCESS);
}

\end{lstlisting}


%----------------------------------------------------------------------------------------------------------------------------------------------------


\newpage
\subsection{launchInstall}
\paragraph{}
launchInstall. This function launch the installation of the service.

\begin{lstlisting}
int launchInstall(BOOL logfileready)
{
	if (!InstallMyService())
	{
		if (logfileready)
		{
			writeInLogFile("Main = > InstallMyService() failed with : ", GetLastError());
			writeInLogFile("\r\n", ERROR_SUCCESS);
		}
		else
			fprintf(stderr, "Main => InstallMyService() failed with : %d\n", GetLastError());
		return(EXIT_FAILURE);
	}
	if (logfileready)
		writeInLogFile("Main => InstallMyService() succeeded\r\n", ERROR_SUCCESS);
	else
		fprintf(stdout, "Main => InstallMyService() succeeded.\n");
	return (EXIT_SUCCESS);
}
\end{lstlisting}

\subsection{launchSetPath}
\paragraph{}
launchSetPath. This function launch the treatment to set the path.

\begin{lstlisting}
int	launchSetPath(char *path, BOOL logfileready)
{
	if (!setPathToClean(path))
	{
		if (logfileready)
		{
			writeInLogFile("Main = > setPathToClean() failed with : ", GetLastError());
			writeInLogFile("\r\n", ERROR_SUCCESS);
		}
		else
			fprintf(stderr, "Main => setPathToClean() failed with : %d\n", GetLastError());
		return(EXIT_FAILURE);
	}
	if (logfileready)
		writeInLogFile("Main => setPathToClean() succeeded\r\n", ERROR_SUCCESS);
	else
		fprintf(stdout, "Main => setPathToClean() succeeded.\n");
	return (EXIT_SUCCESS);
}
\end{lstlisting}

\newpage
\subsection{launchDelete}
\paragraph{}
launchDelete. This function launch the deletion of the service.

\begin{lstlisting}
int	launchDelete(BOOL logfileready)
{
	if (!DeleteMyService())
	{
		if (logfileready)
		{
			writeInLogFile("Main = > DeleteMyService() failed with : ", GetLastError());
			writeInLogFile("\r\n", ERROR_SUCCESS);
		}
		else
			fprintf(stderr, "Main => DeleteMyService() failed with : %d\n", GetLastError());
		return(EXIT_FAILURE);
	}
	if (logfileready)
		writeInLogFile("Main => DeleteMyService() succeeded\r\n", ERROR_SUCCESS);
	else
		fprintf(stdout, "Main => DeleteMyService() succeeded.\n");
	return (EXIT_SUCCESS);
}
\end{lstlisting}

% ----------------------------------------------------------------------------------------------------------------------------------------------------

\newpage
\subsection{InstallMyService}
\paragraph{}
InstallMyService. This function sets up the service.

\begin{lstlisting}
BOOL			InstallMyService()
{
	char		strDir[MAX_PATH + 1];
	SC_HANDLE	schSCManager;
	SC_HANDLE	schService;

	if (!GetCurrentDirectory(MAX_PATH, strDir))
	{
		writeInLogFile("InstallMyService => GetCurrentDirectory() failed with : ", GetLastError());
		writeInLogFile("\r\n", ERROR_SUCCESS);
		return (FALSE);
	}
	lstrcat(strDir, "\\"SERVICE_BIN_NAME);

	if ((schSCManager = OpenSCManager(NULL, NULL, SC_MANAGER_ALL_ACCESS)) == NULL)
	{
		writeInLogFile("InstallMyService => OpenSCManager() failed with : ", GetLastError());
		writeInLogFile("\r\n", ERROR_SUCCESS);
		return (FALSE);
	}

	schService = CreateService(schSCManager, SERVICE_NAME, SERVICE_DESCRIPTOR,
		SERVICE_ALL_ACCESS, SERVICE_WIN32_OWN_PROCESS, SERVICE_DEMAND_START, SERVICE_ERROR_NORMAL,
		(LPCSTR)strDir, NULL, NULL, NULL, NULL, NULL);

	if (schService == NULL)
	{
		writeInLogFile("InstallMyService => CreateService() failed with : ", GetLastError());
		writeInLogFile("\r\n", ERROR_SUCCESS);
		return (FALSE);
	}

	if (!CloseServiceHandle(schService))
	{
		writeInLogFile("InstallMyService => CloseServiceHandle() failed with : ", GetLastError());
		writeInLogFile("\r\n", ERROR_SUCCESS);
		return (FALSE);
	}
	return (TRUE);
}

\end{lstlisting}

\newpage
\subsection{DeleteMyService}
\paragraph{}
DeleteMyService. This function deletes the service.

\begin{lstlisting}
BOOL			DeleteMyService()
{
	SC_HANDLE	schSCManager;
	SC_HANDLE	hService;

	if ((schSCManager = OpenSCManager(NULL, NULL, SC_MANAGER_ALL_ACCESS)) == NULL)
	{
		writeInLogFile("DeleteMyService => OpenSCManager() failed with : ", GetLastError());
		writeInLogFile("\r\n", ERROR_SUCCESS);
		return (FALSE);
	}
	if ((hService = OpenService(schSCManager, SERVICE_NAME, SC_MANAGER_ALL_ACCESS)) == NULL)
	{
		writeInLogFile("DeleteMyService => OpenService() failed with : ", GetLastError());
		writeInLogFile("\r\n", ERROR_SUCCESS);
		return (FALSE);
	}
	if (!DeleteService(hService))
	{
		writeInLogFile("DeleteMyService => DeleteService() failed with : ", GetLastError());
		writeInLogFile("\r\n", ERROR_SUCCESS);
		return (FALSE);
	}
	if (!CloseServiceHandle(hService))
	{
		writeInLogFile("DeleteMyService => CloseServiceHandle() failed with : ", GetLastError());
		writeInLogFile("\r\n", ERROR_SUCCESS);
		return (FALSE);
	}
	return (TRUE);
}
\end{lstlisting}

\newpage
\subsection{ServiceCtrlHandler}
\paragraph{}
ServiceCtrlHandler. This function is called when an event is received by the service.

\begin{lstlisting}
void WINAPI ServiceCtrlHandler(DWORD Opcode)
{
	switch (Opcode)
	{
	case SERVICE_CONTROL_PAUSE:
		g_ServiceStatus.dwCurrentState = SERVICE_PAUSED;
		break;
	case SERVICE_CONTROL_CONTINUE:
		g_ServiceStatus.dwCurrentState = SERVICE_RUNNING;
		break;
	case SERVICE_CONTROL_STOP:
		g_ServiceStatus.dwWin32ExitCode = 0;
		g_ServiceStatus.dwCurrentState = SERVICE_STOPPED;
		g_ServiceStatus.dwCheckPoint = 0;
		g_ServiceStatus.dwWaitHint = 0;
		if (!SetServiceStatus(g_ServiceStatusHandle, &g_ServiceStatus))
		{
			writeInLogFile("ServiceCtrlHandler => SetServiceStatus() failed with : ", GetLastError());
			writeInLogFile("\r\n", ERROR_SUCCESS);
			return;
		}
		break;
	case SERVICE_CONTROL_INTERROGATE:
		break;
	default:
		break;
	}
}
\end{lstlisting}

\newpage
\subsection{ServiceMain}
\paragraph{}
ServiceMain. This function is the entering point of the service. It starts the ServiceCtrlHandler and calls the autoClean function.

\begin{lstlisting}
void WINAPI ServiceMain(DWORD argc, LPTSTR *argv)
{
	writeInLogFile("ServiceMain => Starting the service\r\n", ERROR_SUCCESS);

	g_ServiceStatus.dwServiceType = SERVICE_WIN32;
	g_ServiceStatus.dwCurrentState = SERVICE_START_PENDING;
	g_ServiceStatus.dwControlsAccepted = SERVICE_ACCEPT_STOP;
	g_ServiceStatus.dwWin32ExitCode = 0;
	g_ServiceStatus.dwServiceSpecificExitCode = 0;

	g_ServiceStatusHandle = RegisterServiceCtrlHandler(SERVICE_NAME, ServiceCtrlHandler);
	if (g_ServiceStatusHandle == (SERVICE_STATUS_HANDLE)0)
	{
		writeInLogFile("ServiceMain => RegisterServiceCtrlHandler() failed with : ", GetLastError());
		writeInLogFile("\r\n", ERROR_SUCCESS);
		return;
	}

	g_ServiceStatus.dwCurrentState = SERVICE_RUNNING;
	g_ServiceStatus.dwCheckPoint = 0;
	g_ServiceStatus.dwWaitHint = 0;
	if (!SetServiceStatus(g_ServiceStatusHandle, &g_ServiceStatus))
	{
		writeInLogFile("ServiceMain => SetServiceStatus() failed with : ", GetLastError());
		writeInLogFile("\r\n", ERROR_SUCCESS);
		return;
	}


	writeInLogFile("ServiceMain => Starting to clean\r\n", ERROR_SUCCESS);
	/*
	** ALGORITHME ICI
	*/
	char *path;

	path = malloc(MAX_PATH * sizeof(char));
	getPathToClean(&path);
	if (path == NULL)
		return;
	if (!autoClean(path))
		return;
	writeInLogFile("ServiceMain => ended without problem\r\n", ERROR_SUCCESS);
}
\end{lstlisting}

% ----------------------------------------------------------------------------------------------------------------------------------------------------

\newpage
\subsection{autoClean}
\paragraph{}
autoClean. This function is the main algorithm of cleaning. It is called recursively in the path given to delete *.tmp or temporary files.

\begin{lstlisting}
BOOL				autoClean(char *path)
{
	WIN32_FIND_DATA	FindFileData;
	HANDLE			hFind;
	char			*fullpath = malloc(MAX_PATH * sizeof(char));

	fullpath = _strdup(path);
	lstrcat(fullpath, "\\*.*");
	if ((hFind = FindFirstFile(fullpath, &FindFileData)) == INVALID_HANDLE_VALUE)
	{
		writeInLogFile("autoClean => FindFirstFile() failed with : ", GetLastError());
		writeInLogFile("\r\n", ERROR_SUCCESS);
		return (FALSE);
	}
	do
	{
		if ((FindFileData.dwFileAttributes & FILE_ATTRIBUTE_DIRECTORY)
			&& (strcmp(FindFileData.cFileName, ".") != 0)
			&& (strcmp(FindFileData.cFileName, "..") != 0))
		{
			char	*newpath = goInto(path, FindFileData.cFileName);
			autoClean(newpath);
		}
		else if (FindFileData.dwFileAttributes & FILE_ATTRIBUTE_TEMPORARY)
		{
			if (!deleteFoundFile(path, FindFileData.cFileName))
				return(FALSE);
		}
		else if (FindFileData.dwFileAttributes & FILE_ATTRIBUTE_ARCHIVE)
		{
			if (isExtensionTMP(FindFileData.cFileName))
				if (!deleteFoundFile(path, FindFileData.cFileName))
					return(FALSE);
		}
	} while (FindNextFile(hFind, &FindFileData) != 0);

	if (!FindClose(hFind))
	{
		writeInLogFile("autoClean => FindClose() failed with : ", GetLastError());
		writeInLogFile("\r\n", ERROR_SUCCESS);
		return (FALSE);
	}

	return (TRUE);
}
\end{lstlisting}

\newpage
\subsection{goInto}
\paragraph{}
goInto. This function is just a cleaner way to build the path to a sub-directory, with the addition of ``\\''.

\begin{lstlisting}
char		*goInto(char *path, char *filename) {
	char	*newpath = malloc(MAX_PATH * sizeof(char));

	newpath = _strdup(path);
	lstrcat(newpath, "\\");
	lstrcat(newpath, filename);

	return (newpath);
}
\end{lstlisting}

\subsection{isExtensionTMP}
\paragraph{}
isExtensionTMP. This function tests if a file has a .tmp extension.
\begin{lstlisting}
BOOL		isExtensionTMP(char *filename)
{
	char	ext[5];

	memcpy(ext, &filename[strlen(filename) - 4], 4);
	ext[4] = 0;
	if (strcmp(ext, ".tmp") == 0)
		return (TRUE);
	else
		return (FALSE);
}
\end{lstlisting}

\subsection{deleteFoundFile}
\paragraph{}
deleteFoundFile. This function deletes a file.
\begin{lstlisting}
BOOL	deleteFoundFile(char *path, char *cFileName)
{
	char	*ficToDele = goInto(path, cFileName);

	if (!DeleteFile(ficToDele))
	{
		writeInLogFile("deleteFoundFile => DeleteFile() failed with : ", GetLastError());
		return (FALSE);
	}
	writeInLogFile("deleteFoundFile => \"", ERROR_SUCCESS);
	writeInLogFile(ficToDele, ERROR_SUCCESS);
	writeInLogFile("\" deleted without problem.\r\n", GetLastError());
	return (TRUE);
}
\end{lstlisting}

% ----------------------------------------------------------------------------------------------------------------------------------------------------

\newpage
\subsection{setPathToClean}
\paragraph{}
setPathToClean. This function sets the path to clean in the service registry key.

\begin{lstlisting}
BOOL		setPathToClean(char *path)
{
	HKEY	hKey;

	if (PathFileExists(path))
	{
		if (RegCreateKeyEx(SERVICE_ROOT_KEY, SERVICE_PATH_TO_CLEAN, 0, NULL, REG_OPTION_NON_VOLATILE, KEY_WRITE, NULL, &hKey, NULL) != ERROR_SUCCESS)
		{
			writeInLogFile("setPathToClean => RegCreateKey() failed to create \"HKEY_LOCAL_MACHINE", ERROR_SUCCESS);
			writeInLogFile(SERVICE_PATH_TO_CLEAN, ERROR_SUCCESS);
			writeInLogFile("\" with : ", GetLastError());
			writeInLogFile("\"\r\n", ERROR_SUCCESS);
			return (FALSE);
		}
		if (RegSetValueEx(hKey, "path", 0, REG_SZ, path, strlen(path)) != ERROR_SUCCESS)
		{
			writeInLogFile("setPathToClean => RegSetValueEx() failed with : ", GetLastError());
			writeInLogFile("\r\n", ERROR_SUCCESS);
			return (FALSE);
		}
		if (RegCloseKey(hKey) != ERROR_SUCCESS)
		{
			writeInLogFile("setPathToClean => RegCloseKey() failed with : ", GetLastError());
			writeInLogFile("\r\n", ERROR_SUCCESS);
			return (FALSE);
		}
		writeInLogFile("setPathToClean => The key has been successfully created with path : \"", ERROR_SUCCESS);
		writeInLogFile(path, ERROR_SUCCESS);
		writeInLogFile("\"\r\n", ERROR_SUCCESS);
		return (TRUE);
	}
	else
	{
		writeInLogFile("setPathToClean => \"", ERROR_SUCCESS);
		writeInLogFile(path, ERROR_SUCCESS);
		writeInLogFile("\" is not a valid path, PathFileExists() failed with : ", GetLastError());
		writeInLogFile("\r\n", ERROR_SUCCESS);
		return(FALSE);
	}
}
\end{lstlisting}

\newpage
\subsection{getPathToClean}
\paragraph{}
getPathToClean. This function gets the path from the service registry key.

\begin{lstlisting}
BOOL		getPathToClean(char **path) {
	HKEY	hKey;
	BYTE	buf[255];
	DWORD	dwType;
	DWORD	dwBufSize;

	if (RegOpenKey(SERVICE_ROOT_KEY, SERVICE_PATH_TO_CLEAN, &hKey) == ERROR_SUCCESS)
	{
		dwBufSize = sizeof(buf);
		dwType = REG_SZ;
		if (RegQueryValueEx(hKey, "path", 0, &dwType, buf, &dwBufSize) == ERROR_SUCCESS)
		{
			if (RegCloseKey(hKey) != ERROR_SUCCESS)
			{
				writeInLogFile("getPathToClean => RegCloseKey() failed with : ", GetLastError());
				return (FALSE);
			}
			*path = _strdup(buf);
			printf("");
			return (TRUE);
		}
		else
		{
			writeInLogFile("getPathToClean => RegQueryValueEx() failed with : ", GetLastError());
			return (FALSE);
		}
	}
	else
	{
		writeInLogFile("getPathToClean => RegOpenKey() failed to open \"HKEY_LOCAL_MACHINE", ERROR_SUCCESS);
		writeInLogFile(SERVICE_PATH_TO_CLEAN, ERROR_SUCCESS);
		writeInLogFile("\" with : ", GetLastError());
		writeInLogFile("\"\r\n", ERROR_SUCCESS);
		return(FALSE);
	}
}
\end{lstlisting}

% ----------------------------------------------------------------------------------------------------------------------------------------------------

\newpage
\subsection{writeInLogFile}
\paragraph{}
writeInLogFile. This function's aim is to log what happened during the execution.

\begin{lstlisting}
BOOL		writeInLogFile(char *log, int errorCode)
{
	int		len;
	int		loglen;
	char	err[6];
	char	*newlog;

	loglen = strlen(log);
	newlog = _strdup(log);
	if (errorCode != ERROR_SUCCESS)
	{
		if (_itoa_s(errorCode, err, 6, 10) != ERROR_SUCCESS)
		{
			fprintf(stderr, "writeInLogFile => _itoa_s failed with : %d\n", GetLastError());
			return (FALSE);
		}
		lstrcat(newlog, err);
	}
	loglen = strlen(newlog);

	if (!WriteFile(g_LogFile, newlog, loglen, &len, NULL))
	{
		fprintf(stderr, "writeInLogFile => WriteFile failed with : %d\n", GetLastError());
		return (FALSE);
	}
	return (TRUE);
}
\end{lstlisting}

\restoregeometry

\end{document} 